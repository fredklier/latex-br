\documentclass[a4paper,12]{exam}
\usepackage[right=0.75cm, left=0.75cm, top=0.75cm, bottom=1.5cm]{geometry}
\usepackage[utf8]{inputenc} % para acentos
\usepackage{amsmath, amsfonts, amssymb} %para forrmas matemáticas
\usepackage{graphicx} %pacote para o uso de figuras
\usepackage[portuguese]{babel} %para os rótulos automáticos fiquem em português
\usepackage{adjustbox}
\usepackage{multirow}
\usepackage{multicol}
\usepackage{fourier} %transforma a fonte utilizada no latex
\usepackage{tikz}
\usepackage{tabularx}
\usepackage{chemfig}
\usepackage[version=4]{mhchem}
\usepackage{chemformula}
\usepackage{elements} %para a distribuição eletrônica
\usepackage{xymtexpdf}%PDF Mode
\usepackage{epic,carom} %pacote do colesterol
%\usepackage{stree}
\usepackage{xymtex}
\usepackage{siunitx}
\usepackage{enumitem} %para trocar os rótulos dos itens
\usepackage{mathrsfs}





%pacote para o m de mol
\usepackage{trimclip}
\newcommand{\emeta}{\clipbox{0em -1ex .3em -1ex}{$m$}\clipbox{.3em -1ex 0em -1ex}{$\eta$}}\let\substfont\sffamily 
\renewcommand*\printatom[1]{\ensuremath{\mathsf{#1}}}
%para o m de mol

\author{Fred Klier}
\newcommand{\class}{Química}
\newcommand{\term}{2021}
\newcommand{\examnum}{Exercícios de Química}
\newcommand{\examdate}{\today}
\newcommand{\timelimit}{90 minutos}
\newcommand{\examauthor}{Fred Klier}

\pgfdeclarelayer{background}
\pgfdeclarelayer{main}

\pgfsetlayers{background,main}

\pagestyle{headandfoot}
\firstpageheader{}{}{}
\runningheader{}{}{}
\firstpagefooter{\class}{\examnum\ -Page \thepage\ of \numpages}{\term}
\firstpagefootrule{}
\runningfooter{\class}{\examnum\ -Page \thepage\ of \numpages}{\term}
\runningfootrule{}

\begin{document}

\begin{tikzpicture}[remember picture, overlay] %remember picture permite chamar nodes que não estão no mesmo ambiente tikzpicture e overlay permite passar as margens definidas.

	\node(logo) at (current page.north east) [anchor=north east,xshift=-0.25cm, yshift=-0.5cm] {\includegraphics[width=6cm]{cnsm.png}};
	
	\node(nomealuno) at (logo.north west) [anchor=north east]{{\textbf{Nome:}}{\makebox[11cm]{\hrulefill}\textbf{N$^{\circ}$:}}{\makebox[1cm]{\hrulefill}}};
	
	%\node(nota) at (logo.north west) [anchor=north east,xshift=-0.01cm, yshift=-0.5cm]{\textbf{Nota:}{\makebox[1cm]{\hrulefill}}};
	
	\node(dataprova) at (logo.north west) [anchor=north east,xshift=-0.1cm, yshift=-1cm]{Data de aplicação:{\makebox[.1cm]{}}{\makebox[0.6cm]{\hrulefill}}/{\makebox[0.6cm]{\hrulefill}}/\term};
	
	\node(datadev) at (logo.north west) [anchor=north east,xshift=-0.01cm, yshift=-1.6cm]{Data da devolução:{\makebox[.1cm]{}}{{\makebox[0.6cm]{\hrulefill}}/{\makebox[0.6cm]{\hrulefill}}/\term}};
	
	%\node(valor) at (nota.north west) [anchor=north east,xshift=0cm, yshift=0cm] {\textbf{Valor: 30} {\makebox[1.7cm]{}}};
	
	\node(turma) at (logo.north west) [anchor=north east,xshift=-9.94cm, yshift=-.5cm]{2$^{\circ}$ Ano do Ensino médio};
	
	\node(prova) at (turma.south west) [anchor=north west,xshift=0cm, yshift=-.1cm]{\examnum};
	
	\node(professor) at (datadev.north west) [anchor=north east,xshift=-4.8cm, yshift=0cm]{Professor(a): \examauthor};


\end{tikzpicture}

\vspace{1.2cm}
\rule{18cm}{1pt}
%\fbox{$[substância] =\mathscr{M}=\frac{n_1}{V_{(l)}} =\frac{m_1}{M_1  \ast  V_{(l)}}$} \\

\begin{center}
\begin{large}
\fbox{
$1s^2, 2s^2, 2p^6, 3s^2, 3p^6, 4s^2, 3d^{10}, 4p^6, 5s^2, 4d^{10}, 5p^6, 6s^2, 4f^{14}, 5d^{10}, 6p^6, 7s^2, 5f^{14}, 6d^{10}, 7p^6$
}
\end{large}

\fbox{
ex: $_{6}C=1s^2, 2s^2, 2p^2$
}
\end{center}


\begin{multicols}{2}
	\begin{questions}
		
	  \question[] observe o exemplo: \\
	  \fbox{\ch{
  !(\elconf{Li})( "\chlewis{180.}{Li}" )}} 
  \fbox{\ch{
  !(\elconf{F})( "\chlewis{0.90:180:270:}{F}" )
   }}
   
   Faça as distribuições eletrônicas e a representação da fórmula espacial de Lewis dos elementos abaixo:
	{\renewcommand*\thechoice{\alph{choice}} 
			\renewcommand*\choicelabel{\thechoice)}   
   \begin{choices}
   \choice \ce{^{}_{8}O}
   \choice \ce{^{}_{12}Mg}
   \choice \ce{^{}_{19}K}
   \choice \ce{^{}_{26}Fe}
   \choice \ce{^{}_{92}U}
   \end{choices}
}

\question[10] Observe os elementos abaixo.\\
\ce{^{238}_{92}U} 
\ce{^{60}_{28}Ni^{-2}} 
\ce{^{39}_{19}K^+} 
\ce{^{210}_{83}Bi} \\
\begin{parts}
    \part Escreva o número atômico e a massa atômica para cada um dos elementos.
    \makeemptybox{2cm}
    \part Escreva o número de Prótons, neutros e elétrons de cada um dos elementos.
    \makeemptybox{2cm}
\end{parts}



	\end{questions}	
\end{multicols}
\end{document}
