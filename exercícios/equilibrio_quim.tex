\documentclass[a4paper,12]{exam}
\usepackage[right=0.75cm, left=0.75cm, top=0.75cm, bottom=1.5cm]{geometry}
\usepackage[utf8]{inputenc} % para acentos
\usepackage{amsmath, amsfonts, amssymb} %para forrmas matemáticas
\usepackage{graphicx} %pacote para o uso de figuras
\usepackage[portuguese]{babel} %para os rótulos automáticos fiquem em português
\usepackage{adjustbox}
\usepackage{multirow}
\usepackage{multicol}
\usepackage{fourier} %transforma a fonte utilizada no latex
\usepackage{tikz}
\usepackage{tabularx}
\usepackage{chemfig}
\usepackage[version=4]{mhchem}
\usepackage{chemformula} %fórmula químicas com \ch
\usepackage{elements} %para a distribuição eletrônica.
\usepackage{xymtexpdf}%PDF Mode para moléculas complexas
\usepackage{epic,carom} %pacote do colesterol
\usepackage{xymtex} %desenha fórmulas químicas estruturais
\usepackage{enumitem} %para trocar os rótulos dos itens
\usepackage{siunitx} %para usar unidades do sistema intenacional
\usepackage{mathrsfs} %letras para notações matemática
\usepackage{xfrac} %para ter opções com frações (sfrac)
\usepackage{rotating} %para rodar o texto


\author{Fred Klier}
\newcommand{\class}{Química}
\newcommand{\term}{2021}
\newcommand{\examnum}{Exercícios de Química}
\newcommand{\examdate}{\today}
\newcommand{\timelimit}{90 minutos}
\newcommand{\examauthor}{Fred Klier}

\pgfdeclarelayer{background}
\pgfdeclarelayer{main}

\pgfsetlayers{background,main}

\pagestyle{headandfoot}
\firstpageheader{}{}{}
\runningheader{}{}{}
\firstpagefooter{\class}{\examnum\ - Page \thepage\ of \numpages}{\term}
\firstpagefootrule
\runningfooter{\class}{\examnum\ - Page \thepage\ of \numpages}{\term}
\runningfootrule

\begin{document}

\begin{tikzpicture}[remember picture, overlay] %remember picture permite chamar nodes que não estão no mesmo ambiente tikzpicture e overlay permite passar as margens definidas.

	\node(logo) at (current page.north east) [anchor=north east,xshift=-0.25cm, yshift=-0.5cm] {\includegraphics[width=6cm]{cnsm.png}};
	
	\node(nomealuno) at (logo.north west) [anchor=north east]{{\textbf{Nome:}}{\makebox[11cm]{\hrulefill}\textbf{N$^{\circ}$:}}{\makebox[1cm]{\hrulefill}}};
	
	%\node(nota) at (logo.north west) [anchor=north east,xshift=-0.01cm, yshift=-0.5cm]{\textbf{Nota:}{\makebox[1cm]{\hrulefill}}};
	
	\node(dataprova) at (logo.north west) [anchor=north east,xshift=-0.1cm, yshift=-1cm]{Data de aplicação:{\makebox[.1cm]{}}{\makebox[0.6cm]{\hrulefill}}/{\makebox[0.6cm]{\hrulefill}}/\term};
	
	\node(datadev) at (logo.north west) [anchor=north east,xshift=-0.01cm, yshift=-1.6cm]{Data da devolução:{\makebox[.1cm]{}}{{\makebox[0.6cm]{\hrulefill}}/{\makebox[0.6cm]{\hrulefill}}/\term}};
	
	%\node(valor) at (nota.north west) [anchor=north east,xshift=0cm, yshift=0cm] {\textbf{Valor: 30} {\makebox[1.7cm]{}}};
	
	\node(turma) at (logo.north west) [anchor=north east,xshift=-9.94cm, yshift=-.5cm]{2$^{\circ}$ Ano do Ensino médio};
	
	\node(prova) at (turma.south west) [anchor=north west,xshift=0cm, yshift=-.1cm]{\examnum};
	
	\node(professor) at (datadev.north west) [anchor=north east,xshift=-4.8cm, yshift=0cm]{Professor(a): \examauthor};


\end{tikzpicture}

\vspace{1.2cm}
\rule{18cm}{1pt}
%\fbox{$[substância] =\mathscr{M}=\frac{n_1}{V_{(l)}} =\frac{m_1}{M_1  \ast  V_{(l)}}$} \\

\begin{center}
\begin{Huge}
\fbox{
Exercícios
}
\end{Huge}
\end{center}


\begin{multicols}{2}
	\begin{questions}
		
	
			\question Qual a Concentração de íons sulfato formados em uma solução 			de 1 litro com 0,1 mol de \ce{BaSO4} $(K_s = 1,0 \times 10^{-10})$?
			\begin{oneparchoices}
					\choice $1,0 \times 10^{-2}$
					\choice $1,0 \times 10^{-3}$
					\choice $1,0 \times 10^{-4}$
					\choice $1,0 \times 10^{-5}$
					\choice $1,0 \times 10^{-6}$
			\end{oneparchoices}

			\question Qual a Concentração de íons sulfato formados em uma solução 			de 1 litro com 1 mol de \ce{Ca3(PO4)2} $(K_s = 1,0 \times 10^{-25})$?\\
				\begin{oneparchoices}
						\choice $1,0 \times 10^{-6}$
						\choice $1,0 \times 10^{-5}$
						\choice $1,0 \times 10^{-4}$
						\choice $1,0 \times 10^{-3}$
						\choice $1,0 \times 10^{-2}$
				\end{oneparchoices}

			\question Uma reação química atinge o equilíbrio químico quando:
				\begin{enumerate}[label=\alph*)]
				  \item ocorre simultaneamente nos sentidos direto e inverso.
				  \item as velocidades das reações direta e inversa são iguais.
				  \item os reagentes são totalmente consumidos.
				  \item a temperatura do sistema é igual à do ambiente.
				  \item a razão entre as concentrações de reatantes e produtos é unitária.
				\end{enumerate}

				\question Escreva a expressão da constante de equilíbrio em termos de concentração $(K_c)$ dos seguintes equilíbrios:
				\begin{enumerate}[label=\alph*)]
				  \item \ce{2 NO_{(g)} + O2_{(g)} <--> 2 NO2_{(g)}}
				  \item \ce{PCl5_{(g)} <--> PCl3_{(g)}+ Cl2_{(g)}}
				  \item \ce{4 HCl_{(g)} + O2_{(g)} <--> 2 H2O_{(g)} + 2 Cl2_{(g)}}
				  \item \ce{C_{(s)} + H2O_{(g)} <--> CO_{(g)} + H2_{(g)}}
				  \item \ce{Mg_{(s)} + 2 H^+_{(aq)} <--> Mg^{2+}_{(aq)} + H2_{(g)} }
				  \item \ce{CrO4^{2–}_{(aq)} + 2 H^+_{(aq)} <--> Cr2O7^{2–}_{(aq)} + H2O_{(l)}}
				\end{enumerate}
		
				\question Em determinadas condições de temperatura e pressão, existe 0,5 mol/L de \ce{N2O4} em equilíbrio com 2 mol/L de \ce{NO2}, segundo a equação \ce{N2O4_{(g)} <--> 2 NO2_{(g)}}. Qual o valor da constante $(K_c)$ desse equilíbrio, nas condições da experiência?
				\makeemptybox{2cm}
				\question São colocados 8,0 mol de amônia num recipiente fechado de 5,0 litros de capacidade. Acima de 450 $^{\circ}C$, estabelece-se, após algum tempo, o equilíbrio:
				\begin{center}
				  \ce{2NH3_{(g)} <--> 3H3_{(g)} + N2_{(g)}}
				\end{center}
				Sabendo que a variação do número de mol dos participantes está registrada no gráfico, podemos afirmar que,nestas condições, a constante de equilíbrio, Kc, é igual a:

				falta o gráfico usberco pág 403 ex 6.

				\makeemptybox{2cm}
	\end{questions}	
\end{multicols}
\end{document}
