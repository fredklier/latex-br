\documentclass[a4paper,12]{exam}
\usepackage[right=0.75cm, left=0.75cm, top=0.75cm, bottom=1.5cm]{geometry}
\usepackage[utf8]{inputenc} % para acentos
\usepackage{amsmath, amsfonts, amssymb} %para forrmas matemáticas
\usepackage{graphicx} %pacote para o uso de figuras
\usepackage[portuguese]{babel} %para os rótulos automáticos fiquem em português
\usepackage{adjustbox}
\usepackage{multirow}
\usepackage{multicol}
\usepackage{fourier} %transforma a fonte utilizada no latex
\usepackage{tikz}
\usepackage{tabularx}
\usepackage{chemfig}
\usepackage{isotope}%para escrever os isótopos
\usepackage[version=4]{mhchem} %bioquímica
\usepackage{chemformula} %fórmula químicas
\usepackage{elements} %para a distribuição eletrônica.
\usepackage{xymtexpdf}%PDF Mode para moléculas complexas
\usepackage{epic,carom} %pacote do colesterol
\usepackage{xymtex} %desenha fórmulas químicas estruturais
\usepackage{enumitem} %para trocar os rótulos dos itens
\usepackage{siunitx} %para usar unidades do sistema intenacional
\usepackage{mathrsfs} %letras para notações matemática
\usepackage{colortbl} %para adicionar cor na tabela

\author{Fred Klier}
\newcommand{\class}{Química}
\newcommand{\term}{2021}
\newcommand{\examnum}{Exercícios de Química}
\newcommand{\examdate}{\today}
\newcommand{\timelimit}{90 minutos}
\newcommand{\examauthor}{Fred Klier}

\pgfdeclarelayer{background}
\pgfdeclarelayer{main}

\pgfsetlayers{background,main}

\pagestyle{headandfoot}
\firstpageheader{}{}{}
\runningheader{}{}{}
\firstpagefooter{\class}{\examnum\ - Page \thepage\ of \numpages}{\term}
\firstpagefootrule
\runningfooter{\class}{\examnum\ - Page \thepage\ of \numpages}{\term}
\runningfootrule

\begin{document}

\begin{tikzpicture}[remember picture, overlay] %remember picture permite chamar nodes que não estão no mesmo ambiente tikzpicture e overlay permite passar as margens definidas.

	\node(logo) at (current page.north east) [anchor=north east,xshift=-0.25cm, yshift=-0.5cm] {\includegraphics[width=6cm]{cnsm.png}};
	
	\node(nomealuno) at (logo.north west) [anchor=north east]{{\textbf{Nome:}}{\makebox[11cm]{\hrulefill}\textbf{N$^{\circ}$:}}{\makebox[1cm]{\hrulefill}}};
	
	%\node(nota) at (logo.north west) [anchor=north east,xshift=-0.01cm, yshift=-0.5cm]{\textbf{Nota:}{\makebox[1cm]{\hrulefill}}};
	
	\node(dataprova) at (logo.north west) [anchor=north east,xshift=-0.1cm, yshift=-1cm]{Data de aplicação:{\makebox[.1cm]{}}{\makebox[0.6cm]{\hrulefill}}/{\makebox[0.6cm]{\hrulefill}}/\term};
	
	\node(datadev) at (logo.north west) [anchor=north east,xshift=-0.01cm, yshift=-1.6cm]{Data da devolução:{\makebox[.1cm]{}}{{\makebox[0.6cm]{\hrulefill}}/{\makebox[0.6cm]{\hrulefill}}/\term}};
	
	%\node(valor) at (nota.north west) [anchor=north east,xshift=0cm, yshift=0cm] {\textbf{Valor: 30} {\makebox[1.7cm]{}}};
	
	\node(turma) at (logo.north west) [anchor=north east,xshift=-9.94cm, yshift=-.5cm]{2$^{\circ}$ Ano do Ensino médio};
	
	\node(prova) at (turma.south west) [anchor=north west,xshift=0cm, yshift=-.1cm]{\examnum};
	
	\node(professor) at (datadev.north west) [anchor=north east,xshift=-4.8cm, yshift=0cm]{Professor(a): \examauthor};


\end{tikzpicture}

\vspace{1.2cm}
\rule{18cm}{1pt}
%\fbox{$[substância] =\mathscr{M}=\frac{n_1}{V_{(l)}} =\frac{m_1}{M_1  \ast  V_{(l)}}$} \\


\begin{table}[h]
\centering
\begin{tabular}{c|c|c|c|c|c}
Partícula & número atômico ($Z$) & prótons & elétrons & nêutrons & número de massa ($A$)  \\ 
\hline
A         & ---                  & 83      & 83       & 126      & ---                    \\
D         & ---                  & 55      & 54       & ---      & 133                    \\
E         & 16                   & ---     & 18       & 16       & ---                    \\
G         & ---                  & 56      & 54       & ---      & 137                    \\
J         & 55                   & ---     & 55       & 82       & ---                   
\end{tabular}
\end{table}

\begin{multicols}{2}
  	\begin{questions}
	  \question observe a tabela acima:
	  
Baseado nos dados acima, indique quais são, respectivamente, isótopos e isóbaros entre si:

Isótopos/ Isóbaros:
	  	\makeemptybox{2cm}
	  
	  \question Considere os átomos \ce{^{38}_{20}X}, \ce{^{40}_{}Y} e \ce{^{}_{20}Z}. Sendo Y isótono de X e isóbaro de Z. Demonstre com cálculos o número atômico de Y, a massa de Z e faça a distribuição eletrônica de \ce{^{}_{}Y^{2+}} 
	  	\makeemptybox{2cm}
	  	
	  \question Os átomos M e N são isóbaros e apresentam as seguintes características:
	  \begin{center}
	  \ce{^{5x}_{10+x}M} \ce{^{4x+8}_{11+x}N}
	  \end{center}
	  
	  Demonstre com cálculos os números atômicos e os
	  números de massa de M e N. .	\makeemptybox{2cm}
	  
	  \question X é isótopo de \ce{^{41}_{20}Ca} e isótono de \ce{^{41}_{19}K}. Demonstre com cálculos o número de massa de X. 
	  	\makeemptybox{2cm}
	  
	  
	 \end{questions}
\end{multicols}
\end{document}
