\documentclass[a4paper,12]{exam}
\usepackage[right=0.75cm, left=0.75cm, top=0.75cm, bottom=1.5cm]{geometry}
\usepackage[utf8]{inputenc} % para acentos
\usepackage{amsmath, amsfonts, amssymb} %para forrmas matemáticas
\usepackage{graphicx} %pacote para o uso de figuras
\usepackage[portuguese]{babel} %para os rótulos automáticos fiquem em português
\usepackage{adjustbox}
\usepackage{multirow}
\usepackage{multicol}
\usepackage{fourier} %transforma a fonte utilizada no latex
\usepackage{tikz}
\usepackage{tabularx}
\usepackage{chemfig}
\usepackage[version=4]{mhchem}
\usepackage{chemformula}
\usepackage{xymtexpdf}%PDF Mode
\usepackage{epic,carom} %pacote do colesterol
\usepackage{xymtex}
\usepackage{mathrsfs}


\author{Fred Klier}
\newcommand{\class}{Química}
\newcommand{\term}{2021}
\newcommand{\examnum}{Exercícios de Química}
\newcommand{\examdate}{\today}
\newcommand{\timelimit}{90 minutos}
\newcommand{\examauthor}{Fred Klier}

\pgfdeclarelayer{background}
\pgfdeclarelayer{main}

\pgfsetlayers{background,main}

\pagestyle{headandfoot}
\firstpageheader{}{}{}
\runningheader{}{}{}
\firstpagefooter{\class}{\examnum\ - Page \thepage\ of \numpages}{\term}
\firstpagefootrule
\runningfooter{\class}{\examnum\ - Page \thepage\ of \numpages}{\term}
\runningfootrule

\begin{document}

\begin{tikzpicture}[remember picture, overlay] %remember picture permite chamar nodes que não estão no mesmo ambiente tikzpicture e overlay permite passar as margens definidas.

	\node(logo) at (current page.north east) [anchor=north east,xshift=-0.25cm, yshift=-0.5cm] {\includegraphics[width=6cm]{cnsm.png}};
	
	\node(nomealuno) at (logo.north west) [anchor=north east]{{\textbf{Nome:}}{\makebox[11cm]{\hrulefill}\textbf{N$^{\circ}$:}}{\makebox[1cm]{\hrulefill}}};
	
	%\node(nota) at (logo.north west) [anchor=north east,xshift=-0.01cm, yshift=-0.5cm]{\textbf{Nota:}{\makebox[1cm]{\hrulefill}}};
	
	\node(dataprova) at (logo.north west) [anchor=north east,xshift=-0.1cm, yshift=-1cm]{Data de aplicação:{\makebox[.1cm]{}}{\makebox[0.6cm]{\hrulefill}}/{\makebox[0.6cm]{\hrulefill}}/\term};
	
	\node(datadev) at (logo.north west) [anchor=north east,xshift=-0.01cm, yshift=-1.6cm]{Data da devolução:{\makebox[.1cm]{}}{{\makebox[0.6cm]{\hrulefill}}/{\makebox[0.6cm]{\hrulefill}}/\term}};
	
	%\node(valor) at (nota.north west) [anchor=north east,xshift=0cm, yshift=0cm] {\textbf{Valor: 30} {\makebox[1.7cm]{}}};
	
	\node(turma) at (logo.north west) [anchor=north east,xshift=-9.94cm, yshift=-.5cm]{2$^{\circ}$ Ano do Ensino médio};
	
	\node(prova) at (turma.south west) [anchor=north west,xshift=0cm, yshift=-.1cm]{\examnum};
	
	\node(professor) at (datadev.north west) [anchor=north east,xshift=-4.8cm, yshift=0cm]{Professor(a): \examauthor};


\end{tikzpicture}

\vspace{1.2cm}
\rule{18cm}{1pt}
%\fbox{$[substância] =\mathscr{M}=\frac{n_1}{V_{(l)}} =\frac{m_1}{M_1  \ast  V_{(l)}}$} \\

\begin{center}
\begin{Huge}
\fbox{
Exercício
}
\end{Huge}
\end{center}


\begin{multicols}{2}
	\begin{questions}
		
	\question Nas reações a seguir, não balanceadas, calcule o NOX de cada elemento e participante e indique quais delas são de oxirredução.
    \begin{parts}
        \part $H_2 + O_2 \rightarrow H_2O$\makeemptybox{0.5cm}
        \part $C + O_2 \rightarrow CO_2$\makeemptybox{0.5cm}
        \part $Fe_2O_3 + CO \rightarrow Fe + CO_2$\makeemptybox{0.5cm}
        \part $HCl + NaOH \rightarrow NaCl + H2O$\makeemptybox{0.5cm}
        \part $CaCO3 \rightarrow CaO + CO2 $\makeemptybox{0.5cm}    
    \end{parts}
		
\question Após o balanceamento das reações abaixo, indique quem é o agente redutor e o agente oxidante:
        \begin{parts}
            \part $Fe_2O_3 + C \rightarrow Fe + CO_2$    
Agente Oxidante:\enspace\hrulefill\\
Agente Redutor:\enspace\hrulefill
            \part $H_3AsO_3 + HIO_3 + HCl \rightarrow ICl + H_3As_O4 + H_2O$
Agente Oxidante:\enspace\hrulefill\\
Agente Redutor:\enspace\hrulefill
            \part $KMnO_4 + FeSO_4 + H_2SO_4 \rightarrow MnSO_4 + Fe_2(SO_4)_3$    
Agente Oxidante:\enspace\hrulefill\\
Agente Redutor:\enspace\hrulefill
            \part $Fe + CuSO_4 \rightarrow Fe_2(SO_4)_3 + Cu$     
Agente Oxidante:\enspace\hrulefill\\
Agente Redutor:\enspace\hrulefill
        \end{parts}

\question[10]Calcule a variação dos potenciais de redução para os pares abaixo:
        \begin{parts}
        \part $Li^+ + e^- \rightarrow Li \ E^{\circ}=-3,05 \ V \ || \ Be^{2+} + 2e^- \rightarrow  Be \ E^{\circ}= \ -1,85 \ V$
    \makeemptybox{0.5cm}
    
        \part $O_2 + 2 \ H_2O + 4\ e^- \rightarrow 4\ OH^-\ E^{\circ}=\ +0,40\ V\ ||\  Hg_2^{2+} + 2 e^- \rightarrow  2\ Hg\ E^{\circ}=\ +0,85\ V$
    \makeemptybox{0.5cm}
    
        \part $SO_4^{2-} + 4 \ H^+ + 2\ e^- \rightarrow SO_{2(g)} + 2\ H_2O\ E^{\circ}=\ +0,20\ V \ ||\ Mg^{2+} + 2\ e^- \rightarrow  Mg_(s)\ E^{\circ}=\ -2,37\ V$
    \makeemptybox{0.5cm}
    
        \part $Au^{3+} + 3 e^- \rightarrow Au\ E^{\circ}=+1,50 V\ ||\  2\ H^+ + 2\ e^- \rightarrow H_2\ E^{\circ}=\ 0,00\ V$
    \makeemptybox{0.5cm}
    
        \end{parts}
		
	\end{questions}	
\end{multicols}
\end{document}
