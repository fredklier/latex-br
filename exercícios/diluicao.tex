\documentclass[a4paper,12]{exam}
\usepackage[right=0.75cm, left=0.75cm, top=0.75cm, bottom=1.5cm]{geometry}
\usepackage[utf8]{inputenc} % para acentos
\usepackage{amsmath, amsfonts, amssymb} %para forrmas matemáticas
\usepackage{graphicx} %pacote para o uso de figuras
\usepackage[portuguese]{babel} %para os rótulos automáticos fiquem em português
\usepackage{adjustbox}
\usepackage{multirow}
\usepackage{multicol}
\usepackage{fourier} %transforma a fonte utilizada no latex
\usepackage{tikz}
\usepackage{tabularx}
\usepackage{chemfig}
\usepackage{isotope}%para escrever os isótopos
\usepackage[version=4]{mhchem} %bioquímica
\usepackage{chemformula} %fórmula químicas
\usepackage{elements} %para a distribuição eletrônica.
\usepackage{xymtexpdf}%PDF Mode para moléculas complexas
\usepackage{epic,carom} %pacote do colesterol
\usepackage{xymtex} %desenha fórmulas químicas estruturais
\usepackage{enumitem} %para trocar os rótulos dos itens
\usepackage{siunitx} %para usar unidades do sistema intenacional
\usepackage{mathrsfs} %letras para notações matemática


\author{Fred Klier}
\newcommand{\class}{Química}
\newcommand{\term}{2021}
\newcommand{\examnum}{Exercícios de Química}
\newcommand{\examdate}{\today}
\newcommand{\timelimit}{90 minutos}
\newcommand{\examauthor}{Fred Klier}

\pgfdeclarelayer{background}
\pgfdeclarelayer{main}

\pgfsetlayers{background,main}

\pagestyle{headandfoot}
\firstpageheader{}{}{}
\runningheader{}{}{}
\firstpagefooter{\class}{\examnum\ - Page \thepage\ of \numpages}{\term}
\firstpagefootrule
\runningfooter{\class}{\examnum\ - Page \thepage\ of \numpages}{\term}
\runningfootrule

\begin{document}

\begin{tikzpicture}[remember picture, overlay] %remember picture permite chamar nodes que não estão no mesmo ambiente tikzpicture e overlay permite passar as margens definidas.

	\node(logo) at (current page.north east) [anchor=north east,xshift=-0.25cm, yshift=-0.5cm] {\includegraphics[width=6cm]{cnsm.png}};
	
	\node(nomealuno) at (logo.north west) [anchor=north east]{{\textbf{Nome:}}{\makebox[11cm]{\hrulefill}\textbf{N$^{\circ}$:}}{\makebox[1cm]{\hrulefill}}};
	
	%\node(nota) at (logo.north west) [anchor=north east,xshift=-0.01cm, yshift=-0.5cm]{\textbf{Nota:}{\makebox[1cm]{\hrulefill}}};
	
	\node(dataprova) at (logo.north west) [anchor=north east,xshift=-0.1cm, yshift=-1cm]{Data de aplicação:{\makebox[.1cm]{}}{\makebox[0.6cm]{\hrulefill}}/{\makebox[0.6cm]{\hrulefill}}/\term};
	
	\node(datadev) at (logo.north west) [anchor=north east,xshift=-0.01cm, yshift=-1.6cm]{Data da devolução:{\makebox[.1cm]{}}{{\makebox[0.6cm]{\hrulefill}}/{\makebox[0.6cm]{\hrulefill}}/\term}};
	
	%\node(valor) at (nota.north west) [anchor=north east,xshift=0cm, yshift=0cm] {\textbf{Valor: 30} {\makebox[1.7cm]{}}};
	
	\node(turma) at (logo.north west) [anchor=north east,xshift=-9.94cm, yshift=-.5cm]{2$^{\circ}$ Ano do Ensino médio};
	
	\node(prova) at (turma.south west) [anchor=north west,xshift=0cm, yshift=-.1cm]{\examnum};
	
	\node(professor) at (datadev.north west) [anchor=north east,xshift=-4.8cm, yshift=0cm]{Professor(a): \examauthor};


\end{tikzpicture}

\vspace{1.2cm}
\rule{18cm}{1pt}

\begin{multicols}{2}
  	\begin{questions}
	  \question em recipiente contendo um líquido incolor e homogêneo que ao utilizar aparelhos foi constatado que conduz corrente elétrica. O líquido deveria ser uma solução diluída de glicose e frutose. Sabendo disso explique se é possível determinar se a solução está contaminada somente com esses dados? 
	  \fillwithlines{8em}
	  
	  \question As soluções químicas são amplamente utilizadas tanto em nosso cotidiano como em
	  laboratórios. Uma delas, solução aquosa de sulfato de cobre, CuSO4, a 1\%, é aplicada no controle
	  fitossanitário das plantas atacadas por determinados fungos. A massa de sulfato de cobre,
	  CuSO4, em gramas, necessária para prepararmos 20 litros dessa solução a 1\% p/V é:
	  \makeemptybox{2cm}
	  
	  \question As soluções químicas são amplamente utilizadas tanto em nosso cotidiano como em
laboratórios. Uma delas, solução aquosa de sulfato de cobre penta-hidratado, \ce{CuSO4.5H2O}, a 1\%, é usado como reagente analítico. De posse de uma solução estoque de 1 mol/L com densidade de 2,0 g/ml quanto deve ser usado de solução estoque para prepararmos 20 litros dessa solução a 1\% m/m é: (dados: Cu=64; S=32; O=16; H=1)
	  \makeemptybox{2cm}

	\question O consumo de água com mais de 10 ppm (partes por milhão) de nitratos
não é recomendável, segundo a Organização Mundial de Saúde. Sabendo-se que a densidade da
água é de aproximadamente 1,0 grama por mililitro, em 1,0 metro cúbico de água (1 000 litros)
a quantidade máxima de nitratos, aceitável pela OMS, seria de:
	
	\question A 20$^{\circ}$C uma solução aquosa de hidróxido de sódio tem uma densidade de 1,04 \si{g/cm^3} e
é 0,946 mol/L em NaOH. A quantidade de matéria e a massa de hidróxido de sódio presentes em 50,0 \si{cm^3}
dessa solução são, respectivamente:\makeemptybox{2cm}
	
	\question Considere as seguintes soluções:
	\begin{enumerate}[label=\Roman*-]
		\item 10 g de NaCl em 100 g de água.
% 9%mm 
		\item 10 g de NaCl em 100 ml de água.
		\item 20 g de NaCl em 180 g de água.
% 10%mm
		\item 10 mols de NaCl em 90 mols de água.
	\end{enumerate}
	Foi perguntado qual das soluções tem concentração 10\% em massa de cloreto de sódio. E um aluno respondeu que era o item I. Diga se ele está certo ou errado e justifique a sua resposta com números. \makeemptybox{2cm}
	
	
	 \end{questions}
\end{multicols}
\end{document}
