\documentclass[a4paper,12,addpoints]{exam}
\usepackage[right=0.75cm, left=0.75cm, top=0.75cm, bottom=1.5cm]{geometry}
\usepackage{amsmath, amsfonts, amssymb} %para forrmas matemáticas
\usepackage{graphicx} %pacote para o uso de figuras
\usepackage[portuguese]{babel} %para os rótulos automáticos fiquem em português
\usepackage{adjustbox}
\usepackage{multirow}
\usepackage{multicol}
\usepackage{tikz}
\usepackage{tabularx}
\usepackage{chemfig}
\usepackage{isotope}%para escrever os isótopos
\usepackage[version=4]{mhchem} %fórmulas químicas e reações químicas com \ce
\usepackage{chemformula} %fórmula químicas com \ch 
\usepackage{elements} %para a distribuição eletrônica.
\usepackage{xymtexpdf}%PDF Mode para moléculas complexas
\usepackage{epic,carom} %pacote do colesterol
\usepackage{xymtex} %desenha fórmulas químicas estruturais
\usepackage{enumitem} %para trocar os rótulos dos itens
\usepackage{siunitx} %para usar unidades do sistema intenacional
\usepackage{mathrsfs} %letras para notações matemática
\usepackage{xfrac} %para ter opções com frações (sfrac)
\usepackage{rotating} %para rodar o texto




\let\substfont\sffamily
\renewcommand*\printatom[1]{\ensuremath{\mathsf{#1}}}


\author{Fred Klier}
\newcommand{\class}{Química}
\newcommand{\term}{2020}
\newcommand{\examnum}{Prova II de Química}
\newcommand{\examdate}{\today}
\newcommand{\timelimit}{90 minutos}
\newcommand{\examauthor}{Fred Klier}

\pgfdeclarelayer{background}
\pgfdeclarelayer{main}

\pgfsetlayers{background,main}

\pagestyle{headandfoot}
\firstpageheader{}{}{}
\runningheader{}{}{}
\firstpagefooter{\class}{\examnum\ - Page \thepage\ of \numpages}{\term}
\firstpagefootrule
\runningfooter{\class}{\examnum\ - Page \thepage\ of \numpages}{\term}
\runningfootrule

%parte que faz o formulário
\newdimen\longline
\longline=\textwidth\advance\longline-4cm

\def\LayoutTextField#1#2{#2} % override default in hyperref

\def\lbl#1{\hbox to 4cm{#1\dotfill\strut}}%
\def\labelline#1#2{\lbl{#1}\vbox{\hbox{\TextField[name=#1,width=#2]{\null}}\kern2pt\hrule}}

\def\q#1{\hbox to \hsize{\labelline{#1}{\longline}}\vskip1.4ex}
% fim da parte do formulário

\begin{document}

\begin{tikzpicture}[remember picture, overlay] %remember picture permite chamar nodes que não estão no mesmo ambiente tikzpicture e overlay permite passar as margens definidas.

	\node(logo) at (current page.north east) [anchor=north east,xshift=-0.25cm, yshift=-0.5cm] {\includegraphics[width=6cm]{cnsm.png}};
	
	\node(nomealuno) at (logo.north west) [anchor=north east]{{\textbf{Nome:}}{\makebox[11cm]{\hrulefill}\textbf{N$^{\circ}$:}}{\makebox[1cm]{\hrulefill}}};
	
	\node(nota) at (logo.north west) [anchor=north east,xshift=-0.01cm, yshift=-0.5cm]{\textbf{Nota:}{\makebox[1cm]{\hrulefill}}};
	
	\node(dataprova) at (logo.north west) [anchor=north east,xshift=-0.01cm, yshift=-1cm]{Data da prova:{\makebox[.79cm]{}}{\makebox[0.6cm]{\hrulefill}}/{\makebox[0.6cm]{\hrulefill}}/\term};
	
	\node(datadev) at (logo.north west) [anchor=north east,xshift=-0.01cm, yshift=-1.6cm]{Data da devolução:{\makebox[.1cm]{}}{{\makebox[0.6cm]{\hrulefill}}/{\makebox[0.6cm]{\hrulefill}}/\term}};
	
	\node(valor) at (nota.north west) [anchor=north east,xshift=0cm, yshift=0cm] {\textbf{Valor: 30} {\makebox[1.7cm]{}}};
	
	\node(turma) at (nota.north west) [anchor=north east,xshift=-7.94cm, yshift=0cm]{2$^{\circ}$ Ano do Ensino médio};
	
	\node(prova) at (turma.south west) [anchor=north west,xshift=0cm, yshift=-.1cm]{\examnum};
	
	\node(professor) at (datadev.north west) [anchor=north east,xshift=-4.8cm, yshift=0cm]{Professor(a): \examauthor};


\end{tikzpicture}

\vspace{1.1cm}

\begin{table*}[h!]	
	\begin{tabularx}{20cm}{cX}
		\cline{1-2}
		\multirow{8}{*}{\textbf{Atenção:}} & {Preencha o cabeçalho corretamente.} \\
		& {Leia atentamente as questões (em silêncio) e não faça perguntas. A interpretação faz parte da prova.}  \\
		& {Para respostas definitivas use somente caneta (tinta azul ou preta). Respostas a lápis serão anuladas.} \\
		& {Não se levante do lugar. Não peça materiais emprestados e não converse.} \\
		& {Evite rasuras. Não use corretivo.} \\
		& {Questões objetivas rasuradas serão desconsideradas.} \\
		& {O aluno só poderá entregar a avaliação após 40 minutos do início da mesma.} \\
		&A prova tem \timelimit\\\cline{1-2}
	\end{tabularx}
\end{table*}


\begin{multicols}{2}
	\begin{questions}
		
	\question[20] Questão simples.
{\fontsize{0.1cm}{1em}\selectfont This is big!}

\begin{tikzpicture}[scale=1]
\draw [<->] (0,4) -- (0,0) -- (4,0);
\node [left] at (0,4) {$V_{mol*seg^-1}$};
\node [below] at (4,0) {$T_{(seg)}$};
\draw[very thick] (0,0) to [out=70,in=180] (1.5,1.5);
\draw[very thick, green] (0,3.75) to [out=280,in=180] (1.5,1.5);
\draw[very thick, red] (1.5,1.5) -- (4,1.5);
\draw[dotted, thick, gray] (.5,0) -- (.5,2);
\node [below] at (0.5,0) {$t_1$};
\draw[dotted, thick, gray] (1.5,0) -- (1.5,1.5);
\node [below] at (1.5,0) {$t_2$};
\draw[dotted] (0,0) grid [step=.5] (4,4);
\end{tikzpicture}
		
		
	\addpoints
	\makeemptybox{2cm}
		
		\question[20] Questão com mais de uma pergunta
		\noaddpoints % to omit double points count
		\begin{parts}
			\part[5] faça 1.
			\fillwithlines{8em}
			\part[15] faça 2.
			\fillwithdottedlines{8em}
		\end{parts}
	
	
		\addpoints
		\question[10] O elemento que possui $Z=92$ é:
		\begin{multicols}{3}
			\begin{choices}
				\choice H
				\choice O
				\choice F
				\choice Se
				\choice Ba
				\choice Pb
				\choice U
				\choice Pu
			\end{choices}
		\end{multicols}
		
		\question[10] multipla escolha em várias linhas
		\begin{choices}
			\choice 1
			\choice 2
			\choice 3
			\choice 4
			\choice 5
		\end{choices}
		
		\question[10] múltipla escolha em uma linha\\
		\begin{oneparchoices}
			\choice 1
			\choice 2
			\choice 3
			\choice 4
			\choice 5
		\end{oneparchoices}
		
		\question[10] Marque as afirmações verdadeiras.
		\addpoints
		\begin{checkboxes}
			\choice um
			\choice dois
			\choice três
			\choice quatro
		\end{checkboxes}
		
		{%
			\checkboxchar{$\Box$} % changing checkbox style locally
			\question[10] Marque as afirmações verdadeiras.
			\addpoints
			\begin{checkboxes}
				\choice um
				\choice dois
				\choice três 
			\end{checkboxes}
		}%
		
		{%\renewcommand\choicelabel{\alph{choice}} outra opção com letras minúsculas \arabic para números \roman para numeros romanos
		%colocar a primeira letra maiúscula para ficar maiúsculo.
		%\renewcommand\choicelabel{(\thechoice)} colocar entre parênteses, o que colocar em volta do \thechoice vai ficar em volta da opção. 
			% changing choice items style locally
			\renewcommand*\thechoice{\Roman{choice}} 
			\renewcommand*\choicelabel{(\thechoice)}
			%
			\question[10] O elemento que possui $Z=92$ é:
			\begin{multicols}{3}
				\begin{choices}
					\choice H
					\choice O
					\choice F
					\choice S
					\choice Ba
					\choice Pb
					\choice U
					\choice Pu
				\end{choices}
			\end{multicols}
		}%
		
		
		\question[10]
		Explanação sobre\ldots
		\makeemptybox{2cm}
		
		\question[10] Choose\textbf{ exactly one} from the following problems to solve. 
		\begin{parts} 
			\part The length of a rectangular garden is $9$\textit{m} longer than its width. If the garden's perimeter is  $182$\textit{m}, what is the area of the garden in square feet? Make a model to illustrate your answer. 
			\vspace{0.1in}
			\part The difference of two numbers is 3. The difference of the squares of the same two numbers is $51$. Find the two numbers. 
		\end{parts}
		


{%
	% changing choice items style locally
	\renewcommand*\thechoice{\arabic{choice}} 
	\renewcommand*\choicelabel{\thechoice)}
	%
	\question[10] Element with $Z=92$ is:
	\begin{multicols}{2}
		\begin{choices}
			\choice H
			\choice O
			\choice F
			\choice S
			\choice Ba
			\choice Pb
			\choice U
			\choice Pu
		\end{choices}
	\end{multicols}
}%	
%Roman numbers pacote enumitem
\begin{enumerate}[label=(\roman*)]
  		\item Two
        \item Three
        \item Four
    \end{enumerate}
  % ...

% Arabic numbers
\begin{enumerate}[label=\arabic*)]
  		\item Two
        \item Three
        \item Four
    \end{enumerate}
  % ...

% Alphabetical
	\begin{enumerate}[label=\alph*)]
		\item Two
        \item Three
        \item \sfrac{1}{2}
    \end{enumerate}

	\question Escreva a expressão da constante de equilíbrio em termos de concentração $(K_c)$ dos seguintes equilíbrios:
				\begin{enumerate}[label=\alph*)]
				  \item \ce{2 NO_{(g)} + O2_{(g)} <--> 2 NO2_{(g)}}
				  \item \ce{PCl5_{(g)} <--> PCl3_{(g)}+ Cl2_{(g)}}
				  \item \ce{4 HCl_{(g)} + O2_{(g)} <--> 2 H2O_{(g)} + 2 Cl2_{(g)}}
				  \item \ce{C_{(s)} + H2O_{(g)} <--> CO_{(g)} + H2_{(g)}}
				  \item \ce{Mg_{(s)} + 2 H^+_{(aq)} <--> Mg^{2+}_{(aq)} + H2_{(g)} }
				  \item \ce{CrO4^{2–}_{(aq)} + 2 H^+_{(aq)} <--> Cr2O7^{2–}_{(aq)} + H2O_{(l)}}
				\end{enumerate}

\end{questions}
\end{multicols}

\end{document}
