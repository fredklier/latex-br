\documentclass[a4paper,addpoints]{exam}
\usepackage[right=0.75cm, left=0.75cm, top=0.75cm, bottom=1.5cm]{geometry}
\usepackage[utf8]{inputenc} % para acentos
\usepackage{amsmath, amsfonts, amssymb} %para forrmas matemáticas
\usepackage{graphicx} %pacote para o uso de figuras
\usepackage[portuguese]{babel} %para os rótulos automáticos fiquem em português
\usepackage{adjustbox}
\usepackage{multirow}
\usepackage{multicol}
\usepackage{fourier} %transforma a fonte utilizada no latex
\usepackage{tikz}
\usepackage{tabularx}
\usepackage{chemfig}
\usepackage{isotope}%para escrever os isótopos
\usepackage[version=4]{mhchem} %bioquímica
\usepackage{chemformula} %fórmula químicas com \ch 
\usepackage{elements} %para a distribuição eletrônica.
\usepackage{xymtexpdf}%PDF Mode para moléculas complexas
\usepackage{epic,carom} %pacote do colesterol
\usepackage{xymtex} %desenha fórmulas químicas estruturais
\usepackage{enumitem} %para trocar os rótulos dos itens
\usepackage{siunitx} %para usar unidades do sistema intenacional
\usepackage{mathrsfs} %letras para notações matemática
\usepackage{xfrac} %para ter opções com frações (sfrac)
\usepackage{rotating} %para rodar o texto


\let\substfont\sffamily
\renewcommand*\printatom[1]{\ensuremath{\mathsf{#1}}}
\pgfdeclarelayer{background}
\pgfdeclarelayer{main}
\pgfsetlayers{background,main}

%%%%%%%%%%%%%%%%%%%%%%%%%%%%%%%%%%%%%%%%%%%%%%%%%%%%%%%
%dados da prova
\author{Fred Klier}
\newcommand{\class}{Química}
\newcommand{\term}{2020}
\newcommand{\examnum}{Prova II de Química}
\newcommand{\examdate}{\today}
\newcommand{\timelimit}{90 minutos}
\newcommand{\examauthor}{Fred Klier}
%%%%%%%%%%%%%%%%%%%%%%%%%%%%%%%%%%%%%%%%%%%%%%%%%%%%%%%

%cabeçalho e rodapé
\pagestyle{headandfoot}
\firstpageheader{}{}{}
\runningheader{}{}{}
\firstpagefooter{\class}{\examnum\ - Page \thepage\ of \numpages}{\term}
\firstpagefootrule
\runningfooter{\class}{\examnum\ - Page \thepage\ of \numpages}{\term}
\runningfootrule


%%%%%%%%%%%%%%%%%%%%%%%%%%%%%%%%%%%%%%%%%%%%%%%%%%%

%Início do cabeçalho

\begin{document}
	
	\begin{tikzpicture}[remember picture, overlay] %remember picture permite chamar nodes que não estão no mesmo ambiente tikzpicture e overlay permite passar as margens definidas.
	
	\node(logo) at (current page.north east) [anchor=north east,xshift=-0.25cm, yshift=-0.5cm] {\includegraphics[width=6cm]{cnsm.png}};
	
	\node(nomealuno) at (logo.north west) [anchor=north east]{{\textbf{Nome:}}{\makebox[11cm]{\hrulefill}\textbf{Nº:}}{\makebox[1cm]{\hrulefill}}};
	
	\node(nota) at (logo.north west) [anchor=north east,xshift=-0.01cm, yshift=-0.5cm]{\textbf{Nota:}{\makebox[1cm]{\hrulefill}}};
	
	\node(dataprova) at (logo.north west) [anchor=north east,xshift=-0.01cm, yshift=-1cm]{Data da prova:{\makebox[.79cm]{}}{\makebox[0.6cm]{\hrulefill}}/{\makebox[0.6cm]{\hrulefill}}/\term};
	
	\node(datadev) at (logo.north west) [anchor=north east,xshift=-0.01cm, yshift=-1.6cm]{Data da devolução:{\makebox[.1cm]{}}{{\makebox[0.6cm]{\hrulefill}}/{\makebox[0.6cm]{\hrulefill}}/\term}};
	
	\node(valor) at (nota.north west) [anchor=north east,xshift=0cm, yshift=0cm] {\textbf{Valor: \numpoints} {\makebox[1.7cm]{}}};
	
	\node(turma) at (nota.north west) [anchor=north east,xshift=-7.94cm, yshift=0cm]{2º Ano do Ensino médio};
	
	\node(prova) at (turma.south west) [anchor=north west,xshift=0cm, yshift=-.1cm]{\examnum};
	
	\node(professor) at (datadev.north west) [anchor=north east,xshift=-4.8cm, yshift=0cm]{Professor(a): \examauthor};
	
	
	\end{tikzpicture}
	
	\vspace{1.1cm}
	
	\begin{table*}[h!]	
		\begin{tabularx}{19.5cm}{cX} %se eu coloco 20cm fica bom no papel mas ele fala que está muito cheio. overflow \hbox
			\cline{1-2}
			\multirow{9}{*}{\textbf{Atenção:}} & {Preencha o cabeçalho corretamente.} \\
			& {Leia atentamente as questões (em silêncio) e não faça perguntas. A interpretação faz parte da prova.}  \\
			& {Para respostas definitivas use somente caneta (tinta azul ou preta). Respostas a lápis serão anuladas.} \\
			& {Não se levante do lugar. Não peça materiais emprestados e não converse.} \\
			& {Evite rasuras. Não use corretivo.} \\
			& {Questões objetivas rasuradas serão desconsideradas.} \\
			& {O aluno só poderá entregar a avaliação após 40 minutos do início da mesma.} \\
			& Só é permitido o uso da tabela fornecida na prova.\\ 
			&Esta prova possui \numpages\ páginas e \numquestions\ questões.\\
			& O tempo de prova é de \timelimit\\
			 \cline{1-2}
		\end{tabularx}
	\end{table*}

% fim do cabeçalho	

%%%%%%%%%%%%%%%%%%%%%%%%%%%%%%%%%%%%%%%%%%%%%%%%%%%%%	


\begin{multicols}{2} %questões em várias colunas
	
\begin{questions}
\addpoints	
\question[10]


\question[10] Ao analisarmos, na mesma temperatura, três líquidos conhecidos: água, etanol e éter etílico. Obtemos a seguinte tabela:

\begin{tabular}{ |c|c|c| } 
	\hline
	Líquido & Temperatura & Pressão de vapor \\ \hline
	Água & $20^{\circ}C$ & $17,5 \ mmHg$ \\ \hline
	Etanol & $20^{\circ}C$ & $43,9 \ mmHg$ \\ \hline
	Éter etílico & $20^{\circ}C$ & $442,2 \ mmHg$ \\
	\hline
\end{tabular}

A partir das informações fornecidas na tabela, podemos afirmar que no gráfico:
\begin{center}
\begin{tikzpicture}[line width=1pt, scale=1]
	\draw[<->] (0,3) -- (0,0) -- (4,0);
	\draw[xshift=-8mm, red]  (1,0.5) .. controls (3,0.5) and (2.8,2.5) .. (2.8,2.5);
	\node [above] at (1.97,2.5) {$A$};
	\draw[xshift=-5mm]  (1.2,0.5) .. controls (3,0.5) and (3,2.5) .. (3,2.5);
	\node [above] at (2.45,2.5) {$B$};
	\draw [green] (1.2,0.5) .. controls (3,0.5) and (3,2.5) .. (3,2.5); 
	\node [above] at (2.95,2.5) {$C$};
	%\draw[loosely dashed] (2.95,0) -- (2.95,2);
	%\node [below] at (2.95,0) {$T_4$};
	%\draw[loosely dashed] (2.45,0) -- (2.45,2);
	%\node [below] at (2.45,0) {$T_3$};
	\draw[loosely dashed] (1.97,0) -- (1.97,2);
	\node [below] at (1.97,0) {$T_2$};
	%\draw[loosely dashed] (1.52,0) -- (1.52,2);
	%\node [below] at (1.52,0) {$T_1$};
	\draw[loosely dashed] (0,2) -- (1.97,2);
	\node [left] at (0,2) {$P_1$};
	\draw[loosely dashed] (0,1.5) -- (1.97,1.5);
	\node [left] at (0,1.5) {$P_2$};
	\draw[loosely dashed] (0,1) -- (1.97,1);
	\node [left] at (0,1) {$P_3$};
	\node(pressvap) at (0,0) [anchor=south west,xshift=-0.5cm, yshift=0.1cm, rotate=90] {Pressão de vapor};
	\node(temp) at (0,0) [anchor=north west,xshift=0.75cm, yshift=-0.5cm] {Temperatura};	
\end{tikzpicture}
\end{center}
Responda:

{
	\renewcommand*\thechoice{\Roman{choice}} 
	\renewcommand*\choicelabel{\thechoice)}
	\begin{choices}
		\choice Qual das letras no gráfico corresponde a cada substância?
		\choice Qual das substâncias tem maior ponto de ebulição?
		\choice Qual delas tem o menor ponto de ebulição?
	\end{choices}
}

{
\renewcommand*\thechoice{\alph{choice}} 
\renewcommand*\choicelabel{\thechoice)}
%
	Compare a água do mar com água potável em relação:
%	
\begin{choices}
	\choice às volatilidades;
	\choice às temperaturas de ebulição;
	\choice às temperaturas de congelação.
\end{choices}
}


	\question[20] Questão simples.
{\fontsize{0.1cm}{1em}\selectfont This is big!}

\begin{tikzpicture}[scale=1]
	\draw [<->] (0,4) -- (0,0) -- (4,0);
	\node [left] at (0,4) {$V_{mol*seg^-1}$};
	\node [below] at (4,0) {$T_{(seg)}$};
	\draw[very thick] (0,0) to [out=70,in=180] (1.5,1.5);
	\draw[very thick, green] (0,3.75) to [out=280,in=180] (1.5,1.5);
	\draw[very thick, red] (1.5,1.5) -- (3,1.5);
	\draw[dotted, thick, gray] (.5,0) -- (.5,2);
	\node [below] at (0.5,0) {$t_1$};
	\draw[dotted, thick, gray] (1.5,0) -- (1.5,1.5);
	\node [below] at (1.5,0) {$t_2$};
\end{tikzpicture}



\addpoints
\makeemptybox{2cm}

\question[20] Questão com mais de uma pergunta
\noaddpoints % to omit double points count
\begin{parts}
	\part[5] faça 1.
	\fillwithlines{8em}
	\part[15] faça 2.
	\fillwithdottedlines{8em}
\end{parts}


\addpoints
\question[10] O elemento que possui $Z=92$ é:
\begin{multicols}{3}
	\begin{choices}
		\choice H
		\choice O
		\choice F
		\choice S
		\choice Ba
		\choice Pb
		\choice U
		\choice Pu 🐷
	\end{choices}
\end{multicols}

\question[10] multipla escolha em várias linhas
\begin{choices}
	\choice 1
	\choice 2
	\choice 3
	\choice 4
	\choice 5
\end{choices}

\question[10] múltipla escolha em uma linha\\
\begin{oneparchoices}
	\choice 1
	\choice 2
	\choice 3
	\choice 4
	\choice 5
\end{oneparchoices}

\question[10] Marque as afirmações verdadeiras.
\addpoints
\begin{checkboxes}
	\choice um
	\choice dois
	\choice três
	\choice quatro
\end{checkboxes}

{%
	\checkboxchar{$\Box$} % changing checkbox style locally
	\question[10] Marque as afirmações verdadeiras.
	\addpoints
	\begin{checkboxes}
		\choice um
		\choice dois
		\choice três
	\end{checkboxes}
}%

{%\renewcommand\choicelabel{\alph{choice}} outra opção com letras minúsculas \arabic para números \roman para numeros romanos
	%colocar a primeira letra maiúscula para ficar maiúsculo.
	%\renewcommand\choicelabel{(\thechoice)} colocar entre parênteses, o que colocar em volta do \thechoice vai ficar em volta da opção.
	% changing choice items style locally
	\renewcommand*\thechoice{\Roman{choice}}
	\renewcommand*\choicelabel{(\thechoice)}
	%
	\question[10] O elemento que possui $Z=92$ é:
	\begin{multicols}{3}
		\begin{choices}
			\choice H
			\choice O
			\choice F
			\choice S
			\choice Ba
			\choice Pb
			\choice U
			\choice Pu
		\end{choices}
	\end{multicols}
}%


\question[10]
Explanação sobre\ldots
\makeemptybox{2cm}

\question[10] Choose\textbf{ exactly one} from the following problems to solve. 
\begin{parts} 
	\part The length of a rectangular garden is $9$\textit{m} longer than its width. If the garden's perimeter is  $182$\textit{m}, what is the area of the garden in square feet? Make a model to illustrate your answer. 
	\vspace{0.1in}
	\part The difference of two numbers is 3. The difference of the squares of the same two numbers is $51$. Find the two numbers. 
\end{parts}



{%
	% changing choice items style locally
	\renewcommand*\thechoice{\arabic{choice}} 
	\renewcommand*\choicelabel{\thechoice)}
	%
	\question[10] Element with $Z=92$ is:
	\begin{multicols}{2}
		\begin{choices}
			\choice H
			\choice O
			\choice F
			\choice S
			\choice Ba
			\choice Pb
			\choice U
			\choice Pu
		\end{choices}
	\end{multicols}
}%	






\end{questions}
\end{multicols}

\end{document}
