\documentclass{beamer}
%\usepackage{polyglossia}
%\setmainlanguage{portuguese}
\usepackage{tikz,pgfopts, etoolbox, calc, ifxetex, ifluatex} %pacotes para o metropolis


\usetheme[progressbar=frametitle]{metropolis} %gostei do warsaw e PaloAlto 
\setbeamertemplate{frame numbering}[fraction] % as 6 linhas abaixo são por causa do metropolis.
\useoutertheme{metropolis}
\useinnertheme{metropolis}
\usefonttheme{metropolis}
\usecolortheme{spruce}
\setbeamercolor{background canvas}{bg=white}


%\definecolor{awesome}{rgb}{1.0, 0.13, 0.32} % era para ser vermelho. mas ficou branco
%\definecolor{cnsm}{rgb}{125,5,25} %cor personalizada
%\usecolortheme[named=cnsm]{structure} % comando para cor pesonalizada.
%\usecolortheme{beetle} %use este ou os dois de cima

\title[título curto]{Apresentação em Beamer}
\subtitle[subt curto]{Uma nova alternativa}
\author[Fred Klier]{Carlos Frederico Gitsio Klier Teixeira da Silva}
\institute{Colégio Nossa Senhora Medianeira}
\date[dd/mm/yyyy]{\today}



\begin{document}
	\metroset{block=fill}

\begin{frame}
		\maketitle
\end{frame}

	\begin{frame}[t]{Introdução}\vspace{10pt} % o t é para ficar em cima

		\begin{enumerate}
			\item um
			\item dois
		\end{enumerate}
	\end{frame}

\begin{frame}[t]{Teste de aprendizagem}\vspace{10pt} % o t é para ficar em cima

	\begin{block}{um Bloco}
		\vspace{0.5em}
		texto
		\vspace{0.5em}
	\end{block}
	Um \only<1>{\line(1,0){50} \,} \only<2>{\textcolor{magenta}{líder }} \only<3>{líder} é o maior!\\[10pt]
	O \only<1>{\line(1,0){50} \,} \only<2>{\line(1,0){50} \,} \only<3>{\textcolor{magenta}{comandado }} é assim também.
\end{frame}

\begin{frame}[t]{Outro slide}\vspace{10pt} % o t é para ficar em cima

	\begin{enumerate}
		\item um \line(1,0){50}
		\item dois
	\end{enumerate}
\end{frame}

\begin{frame}[t]{opções}\vspace{10pt} % o t é para ficar em cima
	\begin{columns}[onlytextwith] %para colocar colunas
		\columns[0.5\textwidth]
		xxx
		\columns[0.5\textwidth]
		xxx
	\end{columns}
\end{frame}

\end{document}