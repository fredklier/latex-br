\documentclass{beamer}
\usepackage{tikz,pgfopts, etoolbox, calc, ifxetex, ifluatex} %pacotes para o metropolis
\usepackage{amsmath, amsfonts, amssymb} %para forrmas matemáticas
\usepackage{graphicx} %pacote para o uso de figuras
\usepackage[portuguese]{babel} %para os rótulos automáticos fiquem em português
\usepackage{adjustbox}
\usepackage{multirow}
\usepackage{multicol}
\usepackage{tikz}
\usepackage{tabularx}
\usepackage{chemfig}
\usepackage{isotope}%para escrever os isótopos
\usepackage[version=4]{mhchem} %fórmulas químicas e reações químicas com \ce
\usepackage{chemformula} %fórmula químicas com \ch 
\usepackage{elements} %para a distribuição eletrônica.
\usepackage{xymtexpdf}%PDF Mode para moléculas complexas
\usepackage{epic,carom} %pacote do colesterol
\usepackage{xymtex} %desenha fórmulas químicas estruturais
\usepackage{enumitem} %para trocar os rótulos dos itens
\usepackage{siunitx} %para usar unidades do sistema intenacional
\usepackage{mathrsfs} %letras para notações matemática
\usepackage{xfrac} %para ter opções com frações (sfrac)
\usepackage{rotating} %para rodar o texto

\usetheme[progressbar=frametitle]{berkeley} %
\setbeamertemplate{frame numbering}[fraction] % as contagem dos slides.
%\usecolortheme{lily}


\title[Beamer]{Apresentação em Beamer}
\subtitle[Alt]{Uma nova alternativa}
\author[Fred Klier]{Carlos Frederico Gitsio Klier Teixeira da Silva}
\institute{Colégio Nossa Senhora Medianeira}
\date[dd/mm/yyyy]{\today}

\begin{document}

\begin{frame}
\maketitle
\end{frame}
\begin{frame}[t]{Introdução}
	\section{Introdução}
	\begin{enumerate}[label=\alph*)]
	\item um 
	\item dois
	\end{enumerate}
\end{frame}

\begin{frame}[t]{Introdução}
\section{primeira}
\begin{multicols}{2}	
	\begin{enumerate}[label=\alph*)]
	\item um 
	\item dois
	\end{enumerate}
	\columnbreak
	\begin{tikzpicture}[scale=0.5]
	\draw [<->] (0,5) -- (0,0) -- (5,0);
	\node [left] at (0,5) {$P_{(atm)}$};
	\node [below] at (5,0) {$T_{(K)}$};
	\draw[dotted] (0,0) grid [step=.5] (4.5,4.5);
	\draw[very thick] (0.2,0.2) to [out=40,in=240] (2,2);
	\draw[very thick] (2,2) to [out=20,in=270] (4,4);
	\draw[very thick] (2,2) to [out=120,in=300] (1,4);
	\end{tikzpicture}
	\end{multicols}
\end{frame}

\begin{frame}{Definição}
	\section{Definição}
	\begin{block}{definição}
	um momento para se pensar.
	\end{block}
\end{frame}

\begin{frame}{Foto}
\section{Foto}
\begin{multicols}{2}	
	\begin{enumerate}[label=\alph*)]
	\item um 
	\item dois
	\end{enumerate}
	\columnbreak
	\includegraphics[width=4cm]{cnsm.png}
	
	\end{multicols}
\end{frame}

\begin{frame}{itens}

\begin{itemize}
\item uma coisa
\item outra coisa
\end{itemize}
\end{frame}











\end{document}